\documentclass[a4paper,12pt]{report}
%\documentclass{article}
%\usepackage{polyglossia}
%\setdefaultlanguage{russian}
\usepackage[english,russian]{babel}
\usepackage[utf8]{inputenc}
%\usepackage[T2A]{fontenc}
\usepackage{ucs}
%\usepackage{lists}

\usepackage{verbatim}
\usepackage{xspace}
\usepackage{tikz}
\newcommand{\tikzname}{Ti\emph{k}Z\xspace}
% Объявляем переменные для последующего использования в разных рисунках
		\newcounter{numcolumns}
		\newcounter{numrows}
		\newlength{\lengthx}
		\newlength{\lengthy}
\newsavebox{\boxexample}
\newenvironment{fmpage}[2][c] % Окружение для печати кода примеров в рамке
{\begin{lrbox}{\boxexample}\begin{minipage}[#1]{#2}}
{\end{minipage}\end{lrbox}\fbox{\usebox{\boxexample}}\vspace{3pt}}

\usepackage{amssymb}
\usepackage{paralist} % Это пакет для создания компактных списков

\begin{document}
\chapter{Общие сведения.}

Всем известна блестящая история создания \TeX{} и \LaTeX{}. Не будем повторяться. Если
для кого-то эта история прошла в стороне, то предложим почитать хорошую литетаратуру.
Остановимся на тех свойствах \LaTeX{}, которые делают его использование предпочтительным
во множестве случаев. А именно:
\begin{itemize}
\item отделение процедуры форматирования страницы от процесса формирования контента 
позволяет сосредоточиться на содержании текста;
\item получаемый результат всегда представлен в платформонезависимых форматах, что 
позовляет ему предсказуемо выглядеть на любом оборудовании потребителя;
\item исчерпывающая функциональность в таких проявлениях, как нумерация разделов, 
перекрестные ссылки, список литературы, оглавление и тому подобное;
\item использование в документе сложных математических формул (эту сторону мы 
рассматривать почти не будем).
\end{itemize}
Никак нельзя называть \LaTeX{} текстовым редактором. Это даже не текстовый процессор, 
как всеми нами любимые Micrisoft Word или Libreoffice Writer. \LaTeX{} --- это скорее 
программа для верстки. Однако, как станет ясно из этой статью, чаще всего, это именно то,
что нам нужно. 
\section{Элементарный сеанс работы с \LaTeX.}
Сеанс подготовки документа с помощью \LaTeX{} состоит из следующих действий:
\begin{itemize}
\item подготовка исходного текстового файла, содержащего, в том числе, и команды, 
управляющие форматированием текста;
\item \sloppy обработка текста с помощью соответствующей программы (для получения pdf-файла 
чаще всего это будет команда \verb|pdflatex <имя файла>|), далее будем, что не совсем
правильно, называть этот процесс <<\verb|трансляцией|>>;
\item обработка ошибок трансляции выдаваемых в результате выполнения предыдущего 
действия и повторная трансляция до их отсутствия;
\item просмотр полученного результата с помощью программы просмотра (очень хороший 
вариант \verb|zathura|) или вывод его на печать.
\end{itemize}

Конечно, чаще всего для получения хорошего результата требуется пройтись по этому 
алгоритму несколько раз. Но, как правило, удовольствие от качества получаемого  
результата с лихвой компенсирует затраченные усилия.
Кроме того, для получения работающих ссылок требуется не менее двух проходов при
трансляции исходного файла.
\section{Структура команды \LaTeX.}
Команды \LaTeX{} начинаются с обратного слеша и включают в себя одну или несколько 
латинских букв. Все, что не является латинской буквой считается концом команды. Стоящий
после команды пробел также обозначает конец команды и не считывается при трансляции 
текста. Поэтому, если пробел после команды нужен, то нужно каким-то явным образом 
обозначить конец команды. Например, после команды поставить пустой блок с помощью 
фигурных скобок. 

Команды могут быть с аргументами. Необязательные аргументы помещаются через запятую в 
квадратных скобках после команды. Обязательные аргументы помещаются в фигурные скобки
после необязательных аргументов.

Указания \LaTeXу можно давать с помощью предопределенных окружений. Окружения вызываются
с помощью командых скобок следующим образом:\\
\verb|\begin{название окружения} ... \end{название окружения}|

\section{Документ в целом.}

Вообще, как было сказано ранее, использование \LaTeX{} позволяет автору почти не
задумываться над оформлением документа. Обычно текст набирается просто и гладко, с 
использованием любого текстового редактора, а потом уже набранный текст подвергается
форматированию путем вставки в него управляющих конструкций.
Сам документ содержится между \verb|\begin{document}| и \verb|\end{document}|. Этому 
блоку предшествует так называемая преамбула. Именно здесь сосредотачиваются команды, 
отвечающие за форматирование документа в целом. Что имеется ввиду? Это следующие элементы:
\begin{itemize}
\item выбор общего стилевого оформления документа. 
Имеется несколько предопределенных стилей документа, по названию очевидно их назначение. 
\begin{itemize}
\item статья - \verb|\documentclass{article}|;
\item книга - \verb|\documentclass{book}|;
\item письмо - \verb|\documentclass{letter}|;
\item отчет - \verb|\documentclass{report}|;

\end{itemize}
Нужно заметить, что чаще всего мы будем использовать стиль документа <<article>>. Сразу 
же обозначим, что существует более современная модификация этого стиля. Это <<extarticle>>.
Стиль <<extarticle>> во всех случаях лучше <<article>>, который имеет смысл использовать,
только если <<extarticle>> по каким-то причинам недоступен.

В стиле \verb|article| можно использовать следующие основные необязательные параметры:
\begin{itemize}
\item \verb|12pt| --- размер шрифта, если этот параметр не указать, будет совсем маленький,
10pt (в стилевом пакете \verb|extarticle| можно использовать параметр 14pt);
\item \verb|a4paper| --- размер бумаги $210 X 297$мм, может также быть \verb|a5paper| и
многие другие;
\item \verb|landscape| --- портретная ориентация страницы;
\item \verb|twoside| --- печать зеркальных страниц, её альтернатива \verb|oneside|, 
опция, задаваемая по умолчанию для класса \verb|article|;
\item \verb|twocolumn| --- набор в две колонки.
\end{itemize}
\item набор настроек, связанных с национальными особенностями документа, подробно эти
настройки будут рассмотрены в следующем разделе.
\item подключаемые расширения. 
Набор расширений для \LaTeX{} поражает самое смелое воображение. Существуют расширения 
для рисования шахматных досок, нот, химических формул и многое, многое. Обзор разных
расширений хорошо представлен в.
\item геометрические параметры документа, размеры страницы, размер текста на странице,
размеры полей, содержимое колонтитулов.
\end{itemize}
\section{Национальные особенности}
Несмотря на то, что жить мы пытаемся уже в XXI веке, проблемы связанные с национальными
особенностями остаются. Конечно, на первом месте стоит проблема с кодировкой букв
кириллицы. Изначально \TeX|\LaTeX предназначались для работы с 8-битными кодировками. 
Команды \LaTeX, позволяющие корректно работать с русскими документами, размещаются в 
преамбуле документа. Рассмотрим их поподробнее.

\begin{itemize}
\item
\verb|\usepackage[russian]{babel}| -- эта команда подключает пакет, который	вводит в
действие алгоритм переноса русских слов. Кроме того, этот пакет устанавливает русские 
аналоги служебным словам, используемым в документе. Например, вместо английского 
\verb|Chapter|, будет использован русский аналог \verb|Часть| и другие особенности, 
принятые в русской полиграфии, например кавычки <<елочкой>>.
\item
\verb|\usepackage[utf8]{inputenc}| -- пакет \verb|inputenc| задает кодировку входного 
файла. В современном мире, чаще всего, это именно utf8, однако, Windows часто ещё 
использует устаревшую 8-битную кодировку \verb|cp1251|.
\item
\verb|\usepackage{indentfirst}| -- устанавливает принятый в русской традиции отступ
первой строки первом абзаце нового раздела. В английской традиции такой отступ 
отсутствует. Абзацный отступ появляется со второго абзаца раздела.
\end{itemize}

Описанные выше команды --- это минимальный набор настроек для русскоязычных текстов. Нужно 
отметить, что если документ многоязычный, то в квадратных скобках  нужно перечислить 
через запятую используемые в документе языки. Например:
\\ \verb|\usepackage[russian,english]{babel}|.
\section{Элементы автоматизации.}
Даже в рамках этой небольшой статьи, мы будем использовать некоторые элементы автоматизации.
Мы имеем ввиду действия, которые могут облегчить процесс создания документа. Итак, по порядку.
\subsection{Собственные команды.}
Все очень просто. Собственная команда создается следующим образом:\\
\verb|\newcommand{\mycommand}{желаемые действия}|\\
По сути, создание собственной команды --- это макроопределение. То есть, когда транслятор
встречает в исходном тексте созданную нами команду, он заменяет её тем, что мы задали
при ее определении. Команду можно определить в преамбуле, тогда она будет действовать 
во всем документе, или в любом другом месте, тогда она будет действовать внутри группы, 
то есть внутри фигурных скобок. Для переопределения существующей команды используется
следующая форма:\\
\verb|\renewcommand{\oldcommand}{желаемые действия}|\\
Новые команды могут быть с аргументами. Делается это так:\\
\verb|\newcommand{\mycommand}[n]{желаемые действия}|\\
n -- количество используемых аргументов, в теле макрокоманды места подстановки аргументов
обозначаются как \verb|#1|, \verb|#2| и так далее. Когда будем использовать новую команду,
то после команды аргументы будут передаваться с помощью фигурных скобок. Каждый аргумент
в очередной паре фигурных скобок. Аргументов может быть меньше заявленых, тогда вместо 
недостающих не будет ничего. 

\subsection{Собственные окружения.}
\subsection{Переменные для хранения длин.}
При вёрстке документа, как правило, используется много разных значений длин. Это могут 
быть величины отступов, размеры горизонтальных пробелов, и многое другое. Конкретную 
длину можно запомнить в специально созданной переменной и далее использовать именно эту
переменную. Кроме того существует множество предопределенных длин. Например:
\begin{itemize}
	\item \verb|\textwidth| -- ширина текста на странице;
	\item \verb|\parindent| -- размер отступа первой строки абзаца;
\end{itemize}
{\sloppyСоздать новую переменную, хранящую значение длины можно с помощью команды 
\verb|\newlength{\newname}|. Очевидно, что имя \verb|\newname| должно быть не занято. 
Следующим шагом нужно присвоить новой переменной какое-либо значение. Делается это с 
помощью команды \verb|\setlength{\newname}{new_value}|. Значение переменной, задаваемое 
во вторых фигурных скобках должно быть либо вычисленное умножением коэфициента на одну
из существующих предопределенных переменных или задано с помощью единиц длин, 
используемых в \TeX|\LaTeX.

}
\chapter{Создание документа}
\section{Общие принципы}
В начале этого раздела отметим, что хороший документ не должен иметь вычурного оформления.
Мы будем придерживаться того, что любой элемент офомления должен быть максимально простым.

Чаще всего, набранный в текстовом редакторе текст в документе будет выглядеть, как тот 
же самый текст. Исключение составляют только символы \verb|{, }, $, &, #, %, _, ",^, \|,
которые имеют другое  назначения, а для того, чтобы в тексте появились именно эти символы, 
перед ними нужно поставить обратный слеш \verb|\|. \LaTeX{} при трансляции старается так
расставить промежутки между словами и между строчками, чтобы страница выглядела красиво.
За это мы его и любим. Обычно, результат получается хорошим, а когда не получается, то 
требуется вмешательство человека, выполняющего, как раз, роль верстальщика. В этом
разделе, мы рассмотрим основные приемы форматирования текста.
\section{Страница.}
Есть множество команд для задания параметров страницы. Нужно задать размер бумаги, ширину 
текста, размеры полей и тому подобное. 
Для этих целей удобнее всего использовать пакет \verb|geometry|. Все параметры страницы можно задать, 
используя необязательные параметры пакета \verb|geometry|. 
Приведем список наиболее употребляемых параметров пакета:
\begin{itemize}
\item \sloppy\verb|paper=name|, где \verb|name| может быть одним из этих \verb|a4paper|, 
\verb|a5paper|, \verb|letterpaper|, \verb|a6paper|, \verb|b6paper|, \verb|letterpaper|;  
\item \verb|papersize={ширина,высота}| --- размер бумаги;
\item \verb|landscape| --- альбомная ориентация страницы, по умолчанию \verb|portrait|;
\item \verb|textwidth=значение| --- ширина текста;
\item \verb|textheight=значение| --- высота текста;
\item \verb|lines=число| --- высота текста в строках;
\item \verb|tmargin|,\verb|bmargin|,\verb|lmargin|,\verb|rmargin| --- размеры верхнего, 
нижнего, левого правого полей  соответственно;
\item \verb|twoside| --- режим зеркальных полей.
\end{itemize}
Пример:\\
\verb|\usepackage[a4paper,landscape,tmargin=2cm,lmargin=3cm,|\\
\verb|rmargin=1cm,bmargin=2cm]{geometry}|\\
В этой строке мы определяем размер бумаги А4, альбомную ориентацию страницы
и размеры полей. Очень удобно. 

Сколько строк разместить на странице, определяет \LaTeX, используя для этого приемы, 
которые для пытливого читателя раскрываются в . Очевидно, что разрыв страницы можно 
определить вручную. Для этого используются команды:
\begin{itemize}
\item \verb|\pagebreak| --- эта команда является настойчивой рекомендацией к тому, что
хорошо бы здесь закончить страницу, но \LaTeX, если сильно захочет, может к этой 
рекомендации не прислушаться;
\item \verb|\newpage| --- это прямое указание для разрыва страницы;
\item \verb|\clearpage| --- аналогична предыдущей, но эта команда еще и обязывает \LaTeX{}
напечатать на следующей странице находящиеся в ожидании плавающие объекты (рисунки, 
таблицы и т.п.).
\end{itemize}

Определение колонтитулов. Самый легкий, и чаще всего достаточный, способ определения 
колонтитулов, это воспользоваться одним из предопределенных стилей страницы. Делается 
то с использование команды \verb|\pagestyle{page-style}|. Где \verb|page-style| 
принимает одно из следующих значений:
\begin{itemize}
\item \verb|empty| --- верхний и нижний колонтитулы пусты;
\item \verb|plain| --- номер страницы печатается в нижнем колонтитуле в середине 
страницы, верхний колонтитул пуст;
\item \verb|headings| --- номер страницы печатается в верхнем колонтитуле, кроме того, 
в верхнем же колонтитуле печатается еще некоторая информация, определяемая классом
документа, обычно это заголовок раздела;
\item \sloppy\verb|myheadings| --- содержание верхнего колонтитула командами \verb|markboth| 
и \verb|markright| пользователем.
\end{itemize}
Существует более гибкий способ определения содержимого колонтитулов. Для этого в стилевом
файле потребуется переопределять команды \verb|@evenhead|, \verb|@oddhead|, 
\verb|@evenfoot|, \verb|@oddfoot|. Прочитать об этой технике можно в.
\section{Абзац.}
По умолчанию абзац выравнивается по ширине, если нужно выровнять содержимое абзаца по
левому краю, по центру или по правому краю то используются команды:
\begin{itemize}
\item \verb|\raggedright| --- абзац будет иметь рваный правый край -- это называется
выключка влево;
\item \verb|raggedleft| --- то же самое, но наоборот;
\item \verb|centering| --- выравнивание по центру.
\end{itemize}
или окружения:
\begin{itemize}
\item \verb|\begin{flushleft} ... \end{flushleft}| --- выравнивание влево;
\item \verb|\begin{center} ... \end{center}| --- выравнивание по центру;
\item \verb|\begin{flushright} ... \end{flushright}| --- выравнивание вправо;
\end{itemize}
Обязательно нужно понимать, что в абзацах форматированных описанными способами, слова 
не разрываются для переноса.

Как уже говорилось, \LaTeX{} будет стараться выбрать оптимальные промежутки между 
словами. Если этого не получится, то строка будет вылезать на поля. Обычно, это 
просходит от того, что \LaTeX{} не может найти подходящего способа перенести слово или
какую-то конструкцию. Такого, конечно, допустить нельзя и потребуется вмешательство. 
Что можно сделать? Можно явно указать удачные места для разрыва слова для переноса. 
Для этого слово нужно представить следующим образом: \verb|индус\-три\-али\-зация|. 
Такой конструкцией мы задаем желаемые места переноса слова, при этом, в других местах 
данное слово переноситься не сможет. Для того чтобы задать правило переноса конкретного
слова во всем документе, его нужно задать в преамбуле документа следующим образом:\\
\verb|\hypernation{индус-три-али-зация}|.\\
Для того, чтобы запретить разрыв слова для переноса, это слово нужно поместить в бокс
\verb|\mbox{индустриализация}|.
\section{Таблица.}
\section{Списки.}
Почти всегда в документе присутствуют те или иные списки. Это могут быть ненумерованные
списки, или нумерованные. Списки могут быть одноуровневыми или многоуровневыми. 
Таким образом, это важнейшая часть документа.

Начнём с канонических списков, организацию которых осуществляет \LaTeX{} без использования
тонких настроек и дополнительных пакетов. Сразу оговоримся, в таком виде списки вряд ли
стоит использовать, так как их внешний вид достаточно самобытен. 
Для организации маркированного списка используется окружение:\\
%\parbox
\begin{fmpage}{.8\textwidth}
\verb|\begin{itemize} ... \end{itemize}|
\end{fmpage}
\\
При этом элементы списка предваряются командой \verb|\item|.

Списки могут быть вложенными. На верхнем уровне в качестве маркера используется символ
\verb|\bullet|  ($\bullet$). На втором уровне маркером будет широкое тире ($-$), на третьем уровне ---
символ звездочки ($*$),  на последующих уровнях маркером будет центрированная точка ($\cdot$).

Канонические списки используют для оформления достаточно большие пустые пространства.
Это хорошо смотрится в объемных текстах. В этой статье используются как раз такие
списки. Но в небольших документах, из одной-двух страниц, хотелось бы использовать
более компактные решения. Настройки отступов, используемых по умолчанию, можно изменять,
но это достаточно не тривиальная задача. На помощь приходят дополнительные стилевые 
пакеты. Для примера, слева канонический список, справа набранный с помощью пакета
\verb|paralist|.

\vspace{1ex}
\noindent
\begin{fmpage}[t]{0.45\textwidth}
\begin{itemize}
\item элемент списка;
\item{Вложенный список:
 \begin{itemize}
 \item 1-й элемент вложенного списка;
 \item второй элемент;
 \item последний элемент.
 \end{itemize}   
}
\item продолжение списка верхнего уровня

\end{itemize}
\end{fmpage}
\hfill
\begin{fmpage}[t]{0,45\textwidth}
\begin{compactitem}
	\item 1-й элемент
	\item II уровень
		\begin{compactitem}
			\item элемент вложенного;
			\item следующий;
		\end{compactitem}
	\item последний элемент.
\end{compactitem}
\end{fmpage}
Нумерованные списки создаюся с помощью окружения:\\
\verb|\begin{enumerate} ... \end{enumerate}|\\
В свою очередь компактные нумерованные списки создаются с помощью окружения:\\
\verb|begin{compactenum} ... \end{compactenum}|\\
из стилевого пакета \verb|paralist|. При этом, так же, как и в окружениях формирующих 
маркированные списки, элементы списка обозначаются командой \verb|\item|.
\end{document}

