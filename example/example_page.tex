\documentclass{article}
\usepackage[english,russian]{babel}
\usepackage[a5paper,lmargin=3cm,rmargin=1cm]{geometry}
%\hyphenation{how--to-create--a--ground--truth--data--set--for--computer--vision--using--humanities--data}

\begin{document}
This page aggregates blog entries by people who are writing about \TeX{} and related topics.
How to create a ground truth data set for computer vision using Humanities data

Posted on July 4, 2023 by \LaTeX{} Ninja'ing and the Digital Humanities Feed

Today’s blog post is a teaser for a video class called Computer Vision for Digital Humanities (funded by CLARIAH-AT with the support of BMBWF). The self-learning resource (video lessons plus Jupyter notebooks) is an introduction to Computer Vision methods for Digital Humanists. It addresses some Humanities issues that many typical introductions to computer vision do not cover. This post is an example of such reflection. It gives an insight into the first exercise of the class, filtering a list of metadata to create a ground truth dataset for training a classification algorithm. This blogpost does not contain the actual data (so you stay tuned for the video class!) but discusses the issues which arise when creating a ground truth data set for computer vision using Humanities data. Citation suggestion: Suzana Sagadin \& Sarah Lang, How to create a ground truth data set for computer vision using Humanities data, in \LaTeX{} Ninja Blog, 04.07.2023. https://latex-ninja.com/2023/07/04/how-\-to-\-create-\-a-\-ground-\-truth-\-data-\-set-\-for-\-computer-\-vision-\-using-\-humanities-\-data/ Goal of this session This exerciseread more How to create a ground truth data set for computer vision using Humanities data 
{\sloppy

}
This page aggregates blog entries by people who are writing about \TeX{} and related topics.
How to create a ground truth data set for computer vision using Humanities data

Posted on July 4, 2023 by \LaTeX{} Ninja'ing and the Digital Humanities Feed

Today’s blog post is a teaser for a video class called Computer Vision for Digital Humanities (funded by CLARIAH-AT with the support of BMBWF). The self-learning resource (video lessons plus Jupyter notebooks) is an introduction to Computer Vision methods for Digital Humanists. It addresses some Humanities issues that many typical introductions to computer vision do not cover. This post is an example of such reflection. It gives an insight into the first exercise of the class, filtering a list of metadata to create a ground truth dataset for training a classification algorithm. This blogpost does not contain the actual data (so you stay tuned for the video class!) but discusses the issues which arise when creating a ground truth data set for computer vision using Humanities data. Citation suggestion: Suzana Sagadin \& Sarah Lang, How to create a ground truth data set for computer vision using Humanities data, in \LaTeX{} Ninja Blog, 04.07.2023. https://latex-ninja.com/2023/07/04/how-\-to-\-create-\-a-\-ground-\-truth-\-data-\-set-\-for-\-computer-\-vision-\-using-\-humanities-\-data/ Goal of this session This exerciseread more How to create a ground truth data set for computer vision using Humanities data
{\sloppy 

}
Read this post in context >>
Data Feminism as a Challenge for Digital Huma\-nities?

Posted on July 1, 2023 by \LaTeX{} Ninja'ing and the Digital Humani\-ties Feed

During the annual conference of the DHd Association, the Empo\-wer\-ment Working Group organized a workshop on the topic of Data Feminism in the Digital Humanities (organized by Luise Borek, Nora Probst \& Sarah Lang, technical support: Yael Lämmerhirt)[1]. This short blog post aims to present preliminary results to document the event and raise awareness for this essential topic. Everyone is invited to participate in the project and should contact the Empowerment Working Group if interested. Citation suggestion: Luise Borek*, Elena Suárez Cronauer, Pauline Junginger, Sarah Lang, Karoline Lemke \& Nora Probst, Data Feminism as a Challenge for Digital Humanities? [English version], in \LaTeX{} Ninja Blog, 01.07.2023. https://latex-ninja.com/?p=5068 *All authors contributed equally. Disclaimer: This is a machine-translated version of the original German article (found here), powered by ChatGPT 4. I read over it to make sure there’s nothing wildly inappropriate in there but since terms used are crucial when it comes to this topic, the German version is the one weread more Data Feminism as a Challenge for Digital Humanities? 

\TeX{} \& friends related projects

My involvement with \TeX{} started during my studies of mathematics. In 2002 I started building binaries for TeX Live on the alpha-linux architecture. In 2005 I contacted Debian about packaging TeX Live for Debian, and started packaging it with the first successful upload in January 2006. In 2007 I participated the first time at the BachoTeX (which was also EuroTeX) giving a talk about TeX Live in Debian. During the meeting we had long discussions how to improve the TeX Live infrastructure, which was based on XML files. A few days later I sent a proposal based on the Debian package file format to the TeX Live mailing list, and by end of May the new infrastructure was in place. The rest is history.

With my move to Japan I also got interested in typesetting specifics of CJK languages, another source of inspiration for my work.

\end{document}
